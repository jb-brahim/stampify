\documentclass[12pt,a4paper]{report}
\usepackage[utf8]{inputenc}
\usepackage[T1]{fontenc}
\usepackage[french]{babel}
\usepackage{graphicx}
\usepackage{geometry}
\usepackage{hyperref}
\usepackage{titlesec}
\usepackage{fancyhdr}
\usepackage{listings}
\usepackage{color}
\usepackage{float}
\usepackage{array}

% Configuration de la mise en page
\geometry{hmargin=2.5cm,vmargin=2.5cm}

% Configuration des liens hypertextes
\hypersetup{
    colorlinks=true,
    linkcolor=black,
    filecolor=magenta,      
    urlcolor=blue,
    pdftitle={Rapport de Projet - Stampify},
}

% Configuration des en-têtes et pieds de page
\pagestyle{fancy}
\fancyhead[L]{Rapport de Projet}
\fancyhead[R]{Stampify}
\fancyfoot[C]{\thepage}

% Configuration pour le code
\definecolor{codegreen}{rgb}{0,0.6,0}
\definecolor{codegray}{rgb}{0.5,0.5,0.5}
\definecolor{codepurple}{rgb}{0.58,0,0.82}
\definecolor{backcolour}{rgb}{0.95,0.95,0.92}

\lstdefinestyle{mystyle}{
    backgroundcolor=\color{backcolour},   
    commentstyle=\color{codegreen},
    keywordstyle=\color{magenta},
    numberstyle=\tiny\color{codegray},
    stringstyle=\color{codepurple},
    basicstyle=\ttfamily\footnotesize,
    breakatwhitespace=false,         
    breaklines=true,                 
    captionpos=b,                    
    keepspaces=true,                 
    numbers=left,                    
    numbersep=5pt,                  
    showspaces=false,                
    showstringspaces=false,
    showtabs=false,                  
    tabsize=2
}
\lstset{style=mystyle}

\begin{document}

% Page de garde
\begin{titlepage}
    \begin{center}
        \vspace*{1cm}
        
        \Huge
        \textbf{Rapport de Projet de Fin d'Études}
        
        \vspace{0.5cm}
        \LARGE
        Développement d'une plateforme de fidélisation client digitalisée
        
        \vspace{1.5cm}
        \textbf{Stampify}
        
        \vspace{2cm}
        
        \includegraphics[width=0.4\textwidth]{logo_placeholder.png} % Placeholder pour le logo
        
        \vspace{2cm}
        
        \large
        \textbf{Réalisé par :} \\
        Brahim
        
        \vspace{1cm}
        
        \textbf{Encadré par :} \\
        [Nom de l'encadrant]
        
        \vspace{2cm}
        
        \large
        Année Universitaire : 2024-2025
        
    \end{center}
\end{titlepage}

% Table des matières
\tableofcontents
\newpage

% Liste des figures
\listoffigures
\newpage

% Introduction Générale
\chapter*{Introduction Générale}
\addcontentsline{toc}{chapter}{Introduction Générale}

La fidélisation de la clientèle est un enjeu majeur pour les commerces de proximité. Traditionnellement, cette fidélisation passe par des cartes à tampons en papier. Cependant, ce système présente de nombreux inconvénients : les clients perdent souvent leurs cartes, les oublient, et pour les commerçants, il est impossible d'obtenir des statistiques précises sur le comportement de leurs clients ou de relancer ceux qui ne sont pas venus depuis longtemps.

Dans un monde de plus en plus connecté, la digitalisation de ces processus devient une nécessité. C'est dans ce contexte que s'inscrit le projet \textbf{Stampify}. Stampify est une solution web complète permettant aux commerçants de créer et gérer des cartes de fidélité numériques, et aux clients de collecter des tampons via un simple scan de QR code, sans avoir besoin de télécharger une application mobile lourde, grâce à la technologie PWA (Progressive Web App).

Ce rapport détaille le processus de conception et de réalisation de cette application. Il est structuré en trois chapitres principaux. Le premier chapitre présente le cadre du projet, la problématique et la méthodologie adoptée. Le deuxième chapitre se concentre sur la préparation du projet, incluant l'analyse des besoins, la modélisation et l'architecture technique. Enfin, le troisième chapitre décrit la réalisation technique du projet, découpée en quatre sprints selon la méthodologie Scrum, allant de la mise en place de l'environnement jusqu'au déploiement final.

\newpage

% Chapitre 1
\chapter{Cadre du projet}

\section{Introduction}
Ce chapitre a pour objectif de présenter le contexte général du projet Stampify. Nous commencerons par définir la problématique et l'étude de l'existant, avant de présenter la solution proposée. Nous terminerons par la description de la méthodologie de gestion de projet adoptée pour mener à bien ce travail.

\section{Présentation du projet}

\subsection{Problématique}
Les petits commerces (cafés, snacks, coiffeurs, etc.) utilisent majoritairement des cartes de fidélité en carton. Ce système archaïque pose plusieurs problèmes :
\begin{itemize}
    \item \textbf{Pour le client} : Encombrement du portefeuille, oubli ou perte de la carte, détérioration du papier.
    \item \textbf{Pour le commerçant} : Coût d'impression récurrent, aucune donnée sur la fréquence de visite des clients, impossibilité de contacter les clients, risque de fraude (tampons falsifiés).
    \item \textbf{Écologique} : Gaspillage de papier.
\end{itemize}

La problématique principale est donc : \textit{Comment moderniser le système de fidélisation des petits commerces pour le rendre plus pratique, plus sûr et plus riche en données, tout en restant simple d'utilisation ?}

\subsection{Étude de l'existant}
Il existe déjà des solutions sur le marché, mais elles présentent souvent des freins à l'adoption :
\begin{itemize}
    \item \textbf{Applications mobiles dédiées} : Nécessitent que le client télécharge une application pour chaque commerce ou une grosse application "marketplace" où le petit commerce est noyé dans la masse.
    \item \textbf{Systèmes propriétaires coûteux} : Terminaux de paiement spécifiques ou tablettes dédiées qui sont trop chers pour les petits commerçants.
\end{itemize}

\subsection{Solution proposée}
\textbf{Stampify} se positionne comme une alternative légère et accessible.
\begin{itemize}
    \item \textbf{Architecture Web & PWA} : Pas d'installation obligatoire via les stores. L'application est accessible via un navigateur et installable comme une application native.
    \item \textbf{QR Code Unique} : Le commerçant dispose d'un QR code unique. Le client le scanne pour obtenir son tampon.
    \item \textbf{Tableau de bord commerçant} : Permet de gérer les récompenses, voir les statistiques et gérer son abonnement.
    \item \textbf{Modèle SaaS} : Un abonnement mensuel pour les commerçants, gratuit pour les clients.
\end{itemize}

\section{Méthodologie de gestion de projet}

\subsection{Choix de la méthodologie}
Compte tenu de la nature évolutive des besoins et de la nécessité de livrer rapidement des fonctionnalités testables, nous avons opté pour une méthode \textbf{Agile}. Contrairement au cycle en V classique, l'agilité permet une grande flexibilité et une meilleure adaptation aux retours utilisateurs.

\subsection{Présentation de la méthodologie Scrum}
Nous avons spécifiquement choisi le framework \textbf{Scrum}.
\begin{itemize}
    \item \textbf{Sprints} : Le développement est découpé en itérations courtes (Sprints) de 1 à 2 semaines.
    \item \textbf{Artefacts} :
    \begin{itemize}
        \item \textit{Product Backlog} : Liste priorisée de toutes les fonctionnalités.
        \item \textit{Sprint Backlog} : Liste des tâches pour le sprint en cours.
        \item \textit{Incrément} : Version utilisable du produit à la fin de chaque sprint.
    \end{itemize}
    \item \textbf{Cérémonies} : Sprint Planning, Daily Scrum, Sprint Review, Sprint Retrospective.
\end{itemize}

\section{Conclusion}
Ce premier chapitre nous a permis de poser les bases du projet Stampify en identifiant clairement la problématique des cartes de fidélité papier et en proposant une solution numérique adaptée. L'adoption de la méthodologie Scrum nous assurera un pilotage efficace du développement.

\newpage

% Chapitre 2
\chapter{Préparation du projet}

\section{Introduction}
La phase de préparation est cruciale pour la réussite du projet. Ce chapitre détaille l'analyse des besoins, la structuration des cas d'utilisation, l'organisation de l'équipe Scrum, ainsi que l'environnement technique et l'architecture du système.

\section{Analyse et Spécifications des besoins}

\subsection{Spécification des Besoins fonctionnels}
Les besoins fonctionnels décrivent ce que le système doit faire.
\begin{itemize}
    \item \textbf{Gestion des comptes (Commerçants)} : Inscription, connexion sécurisée (JWT), modification du profil.
    \item \textbf{Gestion des cartes de fidélité} : Personnalisation du nombre de tampons requis, définition de la récompense.
    \item \textbf{Génération de QR Code} : Création d'un QR code unique pour chaque commerce.
    \item \textbf{Scan et Attribution} : Le client scanne le QR code, le système identifie le client et ajoute un tampon.
    \item \textbf{Tableau de bord Admin} : Gestion des utilisateurs, suspension des abonnements, statistiques globales.
    \item \textbf{Interface Client} : Visualisation de la carte, progression, liste des récompenses acquises.
\end{itemize}

\subsection{Spécification des Besoins non fonctionnels}
Les besoins non fonctionnels décrivent comment le système doit se comporter.
\begin{itemize}
    \item \textbf{Sécurité} : Hachage des mots de passe, protection contre les injections, sécurisation des API via JWT.
    \item \textbf{Performance} : Temps de réponse rapide (< 200ms pour l'API), chargement optimisé du frontend (Next.js).
    \item \textbf{Disponibilité} : Architecture robuste capable de supporter plusieurs utilisateurs simultanés.
    \item \textbf{Ergonomie} : Interface intuitive et responsive (Mobile First).
\end{itemize}

\section{Identification et structuration de cas d'utilisation}

\subsection{Identification des acteurs}
\begin{itemize}
    \item \textbf{Administrateur} : Gère la plateforme globale, les abonnements et les utilisateurs.
    \item \textbf{Commerçant (Business Owner)} : Crée sa carte de fidélité, génère son QR code, consulte ses statistiques.
    \item \textbf{Client} : Scanne les QR codes, consulte ses cartes de fidélité.
\end{itemize}

\subsection{Diagramme de cas d'utilisation globale}
\begin{figure}[H]
    \centering
    \includegraphics[width=0.8\textwidth]{diagramme_use_case_global.png}
    \caption{Diagramme de cas d'utilisation global}
\end{figure}
\textit{(Note : Ce diagramme illustre les interactions entre les acteurs et les fonctionnalités principales comme "S'authentifier", "Scanner un QR Code", "Gérer une carte".)}

\section{Pilotage du projet avec Scrum}

\subsection{Équipe et rôle}
\begin{itemize}
    \item \textbf{Product Owner} : Définit la vision du produit et priorise le Backlog.
    \item \textbf{Scrum Master} : Facilite le processus Scrum et élimine les obstacles.
    \item \textbf{Équipe de développement} : Réalise la conception, le développement et les tests (Fullstack).
\end{itemize}

\subsection{Le Backlog du produit}
Le Backlog produit contient l'ensemble des User Stories (US), estimées et priorisées. Exemple :
\begin{itemize}
    \item US1 : En tant que commerçant, je veux m'inscrire pour créer ma carte. (Priorité : Haute)
    \item US2 : En tant que client, je veux scanner un QR code pour gagner un point. (Priorité : Haute)
    \item US3 : En tant qu'admin, je veux voir le nombre total d'inscrits. (Priorité : Moyenne)
\end{itemize}

\subsection{Planification de Sprint}
Le projet est divisé en 4 Sprints :
\begin{itemize}
    \item \textbf{Sprint 1} : Initialisation, Authentification & Base de données.
    \item \textbf{Sprint 2} : Gestion des cartes & QR Codes (Côté Commerçant).
    \item \textbf{Sprint 3} : Expérience Client & Scan.
    \item \textbf{Sprint 4} : Administration, Optimisations & Déploiement.
\end{itemize}

\subsection{Diagramme de Gantt}
\begin{figure}[H]
    \centering
    \includegraphics[width=0.9\textwidth]{diagramme_gantt.png}
    \caption{Diagramme de Gantt prévisionnel}
\end{figure}

\section{Environnement de travail}

\subsection{Environnement matériel}
Le développement a été réalisé sur un poste de travail standard (Windows/Mac), avec des tests sur appareils mobiles (Android/iOS) pour valider la partie PWA et le scan de QR code.

\subsection{Environnement Logiciel}
\begin{itemize}
    \item \textbf{IDE} : Visual Studio Code.
    \item \textbf{Gestion de version} : Git & GitHub.
    \item \textbf{API Testing} : Postman.
    \item \textbf{Design} : Figma (pour les maquettes).
    \item \textbf{Navigateurs} : Chrome, Firefox (outils de développement).
\end{itemize}

\section{Architecture du système}
L'application repose sur une architecture \textbf{MERN} (MongoDB, Express, React/Next.js, Node.js).
\begin{itemize}
    \item \textbf{Backend} : API RESTful développée avec Node.js et Express.
    \item \textbf{Base de données} : MongoDB (NoSQL) pour sa flexibilité avec les documents JSON.
    \item \textbf{Frontend} : Next.js (React Framework) pour le rendu côté serveur (SSR) et l'optimisation SEO, utilisant Tailwind CSS pour le style.
    \item \textbf{Communication} : Les échanges se font via protocole HTTP/HTTPS au format JSON.
\end{itemize}

\begin{figure}[H]
    \centering
    \includegraphics[width=0.8\textwidth]{architecture_systeme.png}
    \caption{Architecture technique MERN}
\end{figure}

\section{Conclusion}
Nous avons défini les besoins fonctionnels et techniques, identifié les acteurs et structuré le projet selon la méthode Scrum. L'architecture technique choisie est moderne et adaptée aux contraintes de performance et de scalabilité.

\newpage

% Chapitre 3
\chapter{Réalisation}

\section*{Introduction}
Ce chapitre présente le déroulement concret du développement à travers les quatre sprints planifiés. Pour chaque sprint, nous détaillerons les User Stories, la modélisation, l'implémentation et les tests effectués.

\section{Développement du Sprint 1 : Fondations & Authentification}

\subsection{User stories}
\begin{itemize}
    \item \textbf{US 1.1} : En tant que visiteur, je veux pouvoir m'inscrire en tant que commerçant.
    \item \textbf{US 1.2} : En tant que commerçant, je veux me connecter à mon compte.
    \item \textbf{US 1.3} : En tant que système, je veux sécuriser les mots de passe et les sessions.
\end{itemize}

\subsection{Modélisation}
\textbf{Diagramme de classe (Extrait)} :
La classe \texttt{BusinessOwner} contient les attributs : \texttt{email}, \texttt{password} (hashé), \texttt{businessName}.
\textbf{Diagramme de séquence} : Authentification (Client envoie crédentials $\rightarrow$ API vérifie $\rightarrow$ API renvoie Token JWT).

\subsection{Implémentation}
\begin{itemize}
    \item Mise en place du serveur Node.js/Express.
    \item Configuration de la connexion MongoDB avec Mongoose.
    \item Création du modèle \texttt{BusinessOwner}.
    \item Implémentation des routes \texttt{/api/auth/signup} et \texttt{/api/auth/login}.
    \item Utilisation de \texttt{bcrypt} pour le hachage et \texttt{jsonwebtoken} pour les tokens.
\end{itemize}

\begin{lstlisting}[language=JavaScript, caption=Extrait du contrôleur d'authentification]
exports.login = async (req, res) => {
  const { email, password } = req.body;
  // Vérification de l'utilisateur
  const user = await BusinessOwner.findOne({ email });
  if (!user) return res.status(400).json({ msg: 'Utilisateur non trouvé' });
  // Vérification du mot de passe
  const isMatch = await bcrypt.compare(password, user.password);
  // Génération du token
  const token = jwt.sign({ id: user._id }, process.env.JWT_SECRET);
  res.json({ token, user });
};
\end{lstlisting}

\subsection{Test}
Tests unitaires des API avec Postman :
\begin{itemize}
    \item Inscription avec email valide $\rightarrow$ Succès (201).
    \item Connexion avec mauvais mot de passe $\rightarrow$ Erreur (400).
    \item Accès route protégée sans token $\rightarrow$ Refus (401).
\end{itemize}

\section{Développement du Sprint 2 : Gestion Commerçant & QR}

\subsection{User stories}
\begin{itemize}
    \item \textbf{US 2.1} : En tant que commerçant, je veux configurer ma carte de fidélité (nombre de tampons, récompense).
    \item \textbf{US 2.2} : En tant que commerçant, je veux obtenir mon QR code unique.
    \item \textbf{US 2.3} : En tant que commerçant, je veux voir mon tableau de bord.
\end{itemize}

\subsection{Modélisation}
Ajout de l'objet \texttt{stampCard} dans le modèle \texttt{BusinessOwner}.
Génération d'un UUID unique pour le \texttt{qrToken} à la création du compte.

\subsection{Implémentation}
\begin{itemize}
    \item Création des routes \texttt{/api/card/update} et \texttt{/api/qr/my}.
    \item Développement du Frontend (Next.js) : Page Dashboard, Formulaire de configuration de carte.
    \item Intégration de la librairie \texttt{qrcode} pour générer l'image QR à partir du token.
\end{itemize}

\subsection{Test}
\begin{itemize}
    \item Modification de la récompense $\rightarrow$ Mise à jour en base de données vérifiée.
    \item Affichage du QR Code $\rightarrow$ Scan test avec une application tierce pour vérifier le contenu (URL/Token).
\end{itemize}

\section{Développement du Sprint 3 : Expérience Client & Scan}

\subsection{User stories}
\begin{itemize}
    \item \textbf{US 3.1} : En tant que client, je veux scanner un QR code pour ajouter un tampon.
    \item \textbf{US 3.2} : En tant que client, je veux voir ma progression.
    \item \textbf{US 3.3} : En tant que système, je veux empêcher la fraude (scan multiple rapide).
\end{itemize}

\subsection{Modélisation}
Création du modèle \texttt{Customer} :
\begin{itemize}
    \item \texttt{deviceId} (Identifiant unique du navigateur/appareil).
    \item \texttt{businessId} (Lien vers le commerce).
    \item \texttt{stamps} (Compteur).
\end{itemize}

\subsection{Implémentation}
\begin{itemize}
    \item Route \texttt{/api/scan/:qrToken} : Logique métier complexe.
    \begin{itemize}
        \item Vérification du token QR.
        \item Identification du client par \texttt{deviceId} (Fingerprinting ou LocalStorage).
        \item Incrémentation du compteur.
        \item Rate Limiting : Blocage si scan < 10 secondes.
    \end{itemize}
    \item Frontend : Page de scan avec lecteur QR caméra (html5-qrcode ou react-qr-reader).
    \item Page "Mes Cartes" pour le client.
\end{itemize}

\subsection{Test}
\begin{itemize}
    \item Scan réussi $\rightarrow$ Compteur incrémenté de 0 à 1.
    \item Scan répété immédiatement $\rightarrow$ Erreur "Veuillez patienter".
    \item Scan d'un QR invalide $\rightarrow$ Gestion d'erreur appropriée.
\end{itemize}

\section{Développement du Sprint 4 : Administration & Finalisation}

\subsection{User stories}
\begin{itemize}
    \item \textbf{US 4.1} : En tant qu'admin, je veux voir la liste des commerces.
    \item \textbf{US 4.2} : En tant qu'admin, je veux suspendre un commerce.
    \item \textbf{US 4.3} : En tant qu'utilisateur, je veux une interface fluide et belle (UI/UX).
\end{itemize}

\subsection{Modélisation}
Modèle \texttt{AdminUser} avec rôle spécifique. Middleware \texttt{adminAuth} pour protéger les routes.

\subsection{Implémentation}
\begin{itemize}
    \item Backend : Endpoints \texttt{/api/admin/*}.
    \item Frontend : Interface Admin dédiée (Tableau de bord super-admin).
    \item UI Polish : Utilisation de Shadcn UI et Tailwind pour un design moderne (Glassmorphism, animations).
    \item PWA : Configuration du \texttt{manifest.json} et Service Workers pour l'installation sur mobile.
\end{itemize}

\subsection{Test}
\begin{itemize}
    \item Test d'intrusion : Tentative d'accès admin avec un compte commerçant $\rightarrow$ Refusé.
    \item Test PWA : Installation sur Android et iOS, fonctionnement hors-ligne (partiel).
    \item Validation finale de tous les flux utilisateurs.
\end{itemize}

\section{Conclusion}
La phase de réalisation a permis de transformer les spécifications en un produit fonctionnel. L'approche itérative par Sprints a permis de valider chaque brique (Auth, Business, Client, Admin) progressivement, assurant la qualité et la robustesse de l'application Stampify.

\newpage

% Conclusion Générale
\chapter*{Conclusion Générale}
\addcontentsline{toc}{chapter}{Conclusion Générale}

Le projet \textbf{Stampify} avait pour ambition de moderniser la fidélisation client pour les petits commerces en remplaçant les cartes papier par une solution numérique accessible et efficace. Au terme de ce travail, nous avons développé une application web complète, fonctionnelle et performante.

Nous avons réussi à :
\begin{itemize}
    \item Mettre en place une architecture robuste basée sur la stack MERN/Next.js.
    \item Développer une interface intuitive pour les commerçants et les clients.
    \item Implémenter un système de scan QR code sécurisé et performant.
    \item Intégrer des fonctionnalités d'administration pour la gestion de la plateforme.
\end{itemize}

Ce projet a également été l'occasion d'approfondir nos compétences techniques (Node.js, React, MongoDB, PWA) et méthodologiques (Scrum, Gestion de projet).

\textbf{Perspectives d'avenir} :
Le projet peut encore évoluer avec de nouvelles fonctionnalités :
\begin{itemize}
    \item Notifications Push pour relancer les clients inactifs.
    \item Géolocalisation pour trouver les commerces partenaires à proximité.
    \item Intégration de paiements pour l'abonnement des commerçants (Stripe).
\end{itemize}

Stampify est donc une base solide pour une solution commerciale viable, répondant à un besoin réel du marché avec une approche technologique moderne.

\newpage

% Bibliographie
\begin{thebibliography}{9}
\addcontentsline{toc}{chapter}{Références bibliographiques}

\bibitem{scrum}
Schwaber, K., \& Sutherland, J. (2020). \textit{Le Guide Scrum}. Scrum.org.

\bibitem{nodejs}
Documentation officielle Node.js. \url{https://nodejs.org/en/docs/}

\bibitem{react}
Documentation officielle React. \url{https://react.dev/}

\bibitem{mongodb}
Documentation officielle MongoDB. \url{https://www.mongodb.com/docs/}

\bibitem{nextjs}
Documentation officielle Next.js. \url{https://nextjs.org/docs}

\bibitem{jwt}
JSON Web Token Introduction. \url{https://jwt.io/introduction}

\bibitem{pwa}
MDN Web Docs - Progressive Web Apps. \url{https://developer.mozilla.org/en-US/docs/Web/Progressive_web_apps}

\end{thebibliography}

\end{document}
